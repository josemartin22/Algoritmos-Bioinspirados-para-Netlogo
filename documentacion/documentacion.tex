%---------------------------------------------------------
\documentclass[a4paper,12pt]{report}
\usepackage[utf8]{inputenc}
\usepackage[spanish]{babel}

%---------------------------------------------------------
\usepackage{graphicx}           %incluir imagenes
\usepackage{verbatim}           %comentarios multilinea
\usepackage{subfig}
\usepackage{vmargin}
\setpapersize{A4}

%---------------------------------------------------------
\begin{comment}
\setmargins{2.5cm}       % margen izquierdo
{1.5cm}                        % margen superior
{16.5cm}                      % anchura del texto
{23.42cm}                    % altura del texto
{10pt}                           % altura de los encabezados
{1cm}                           % espacio entre el texto y los encabezados
{0pt}                             % altura del pie de página
{2cm}                           % espacio entre el texto y el pie de página
\end{comment}

% ########################################################################################
% Inicio del contenido del documento

\begin{document}

    \begin{titlepage}
        \centering
        {\includegraphics[width=0.6\textwidth]{images/ugr.jpeg}\par}    
        \vspace{1.5cm}
        \hrule height 0.5pt
        \vspace{1cm}
        {\scshape\LARGE Trabajo Fin de Grado \par}
        \vspace{0.2cm}
        {\scshape\large Grado en Ingeniería Informática \par}
        \vspace{1cm}
        {\itshape\Large \textbf{Algoritmos bioinspirados para NetLogo}\par}
        \vspace{1cm}
        \hrule height 1.5pt
        \vspace{1cm}
        {\normalsize \textbf{Autor}\par}
        \vspace{0.1cm}
        {\normalsize José Antonio Martín Melguizo\par}
        \vspace{1.5cm}
        {\normalsize \textbf{Directora}\par}
        \vspace{0.1cm}
        {\normalsize Rocío Romero Zaliz\par}
		\vspace{3cm}
		{\includegraphics[width=0.3\textwidth]{images/etsiit.png}\par}
		\vspace{0.5cm}
		{\scshape\large Escuela Técnica Superior de Ingeniería Informática y Telecomunicaciones \par}
		\vspace{0.5cm}
        \large{Granada, Noviembre de 2020}
        \vfill
        
    \end{titlepage}

\clearpage

%---------------------------------------------------------
% Índice
\tableofcontents
%---------------------------------------------------------
    
    
% ########################################################################################

\chapter{Introducción}

El origen del modelado basado en agentes surge con la máquina de Von Neumann, la cual era capaz de reproducirse, a partir de una secuencia de celdas en una cuadrícula. Los primeros modelos (conceptuales) surgen en las décadas de los 70 y 80, como el modelo de segregación de Thomas Schelling. 
\vspace{2mm}\\
Fue en la década de los 90 cuando se expandió su desarrollo, liderado por el estudio de los fenómenos sociales como las migraciones estacionales, transmisión de enfermedades, contaminación e incluso combate.
\vspace{4mm}\\
Las metodologías de modelización clásicas, utilizadas en el ámbito académico, como es la matemática (basadas en funciones continuas, derivables y ecuaciones diferencias entre otras) han sido una buena herramienta para el desarrollo de las destrezas y competencias lógico-matemáticas para la representación y resolución de problemas.
\vspace{2mm}\\
La modelización en el ámbito científico tiene como objetivos la creación de modelos, definidos por conjuntos de elementos que interoperan mediante operaciones matemáticas, que representen y aproximen sistemas reales. Aquí es donde los modelos computaciones tienen un papel decisivo para el desarrollo de este nuevo frentem ya que aportan una enorme capacidad de cálculo y visualización.
\vspace{4mm}\\
Surge la modelización basada en agentes \textit{(Agent-based model, ABM)}. Se trata de simular las acciones e interacciones de agentes autónomos sobre un entorno con el fin de evaluar el efecto de sus acciones en éste. Se utilizan en dominios como la biología, ecología o las ciencias sociales.
\vspace{2mm}\\
Dentro de éstos se encuentran los modelos basados en individuos \textit{(Individual-based model, IBM)} que pueden ser más simples que los ABM y se utilizan para resolver problemas prácticos o específicos de la ingeniería.

\clearpage

\section{Abstract}

Este proyecto surge con la intención de mostrar a estudiantes ajenos al ámbito informático,
concrétamente a estudiantes pertenecientes al ámbito de la ingeniería industrial, aeronáutica, de caminos y también a arquitectos, las bases de los modelos más sencillos de algoritmos bioinspirados (algoritmos genéticos, colonias de hormigas, nube de partículas etc.) así como sus aplicaciones y el correcto uso de los distintos parámetros que acarrean cada uno de ellos.
\vspace{4mm}\\
Además se complementará con un entorno de modelado programable diseñado para simular fenómenos naturales y sociales como es \texttt{NetLogo}. Se resolverán problemas típicos de optimización como TSP y otros propios del ámbito de la ingeniería civil.

\vspace{2cm}

\section{Estado del arte}

Actualmente, el modelado de sistemas basados en individuos es algo tangible para personas no necesariamente ligadas al ámbito de la programación informática.
\vspace{2mm}\\
Con el motivo de reducir el obstáculo del uso de lenguajes de programación como Java, C++ o Fortrán, y a la vez, aunar y simplificar el desarrollo de este tipo de modelos, el personal docente de las universidades y centros de investigación han desarrollado distintos entornos y plataformas para este fin.
\vspace{2mm}\\
Entre las herramientas más relevantes a día de hoy nos encontramos a: \textit{Swarm}, \textit{MASON}, \textit{Repast}, \textit{AnyLogic} y  \textit{NetLogo}. 
\vspace{2mm}\\
A día de hoy, Netlogo es muy recomendable para el ámbito académico y docente, ya que contiene diversas herramientas para mostrar el uso de los IBM y trabajar con ellos de forma sencilla.


\chapter{¿Qué es NetLogo?}


\textit{"NetLogo es un entorno de modelado programable para simular fenómenos naturales y sociales. Fue escrito por Uri Wilensky en 1999 y ha estado en continuo desarrollo desde entonces en el Center for Connected Learning and Computer-Based Modeling".}
\vspace{8mm}\\
Es adecuada tanto para estudiantes y profesores debido a su atractiva interfaz, la cual permite cargar simular rápidamente modelos preinstalados y explorar su comportamiento, como para investigadores que requieran de una herramienta lo suficientemente avanzada.
\vspace{2mm}\\
Es especialmente recomendable para modelado de sistemas complejos que se desarrollan en el tiempo. Se puede dar instrucciones a una gran cantidad de agentes de forma independiente, que pueden coordinarse entre sí.
\vspace{6mm}\\
Dispone de una grande variedad de modelos y simulaciones de diferentes ámbitos del conocimiento (ciencias naturales, sociales, economía y medicina), así como de tutoriales y una densa documentación. No obstante hay presentes varias líneas de investigación que continuan desarrollando modelos para dicha plataforma.
\vspace{2mm}\\
Destacar que es gratis \textit{(open source)}, además de ser un entorno multiplataforma, ya que se ejecuta sobre una máquina virtual Java en su versión de escritorio, pudiendo usarse su otra alternativa en versión web (sigue en desarrollo y no presenta las mismas funcionalidades que la versión de escritorio). 


\section{Conclusión}

Teniendo en cuenta el tiempo limitado del que dispone un curso
académico y las amplias posibilidades que ofrece el desarrollo de sistemas basados en agentes, no es viable ofrecer un curso en profundidad de modelado de éstos a estudiantes no relacionados con el ámbito de la informática.
\vspace{2mm}\\
Ésto junto a las características anteriormente presentadas sobre la plataforma NetLogo, hacen de ésta una atractiva opción a la hora de impartir conocimientos de modelado de agentes e individuos en el ámbito docente. 


\section{Definición del problema}


































\end{document}